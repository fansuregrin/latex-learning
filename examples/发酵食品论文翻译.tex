\documentclass[twocolumn]{ctexart}
\usepackage{hyperref}
\usepackage{authblk}
\usepackage{graphicx}
\usepackage{geometry}
\usepackage{threeparttable}
\usepackage{subcaption}

\geometry{paperheight=8.26772in, paperwidth=11.0236in,left=1.8cm,right=1.8cm,top=2cm,bottom=2cm}
\setlength{\columnsep}{40pt}
\providecommand{\keywords}[1] % 大部分期刊的$\TeX$模板已经自定义了关键词的格式,我们这里使用一个最简单的自定义格式:
{
  \textbf{\textit{关键词: }} #1
}

\title{使用搅拌和均质化来改善新鲜,冷藏和冷冻/解冻羊奶中凝固酸奶的发酵特性和理化特性}
\renewcommand\Affilfont{\itshape\small}
\author[1]{Alline Artigiani Lima Tribst 
	\thanks{Corresponding author.\\E-mail addresses: tribst@unicamp.br, alline.lima.tribst@gmail.com (A.A.L. Tribst).}}%
\author[1]{Luiza Toledo Piza Falcade}
\author[1]{Nathália Silva Carvalho}
\author[2]{Bruno Ricardo de Castro Leite Junior}
\author[3]{Miguel Meirelles de Oliveira}
\affil[1]{Center for Food Studies and Research (NEPA), University of Campinas (UNICAMP), Albert Einstein, 291, 13083-852, Campinas, SP, Brazil}
\affil[2]{Department of Food Technology (DTA), Federal University of Viçosa (UFV), University Campus, 36570-900, Viçosa, MG, Brazil}
\affil[3]{Federal Center for Technological Education Celso Suckow da Fonseca (CEFET-RJ), Voluntários da Pátria, 30, 27600-000, Valença, RJ, Brazil}
\date{}

\begin{document}
\maketitle
\captionsetup{font={small}}
\CTEXsetup[format={\Large\bfseries}]{section}

\section*{文章信息}
\keywords{羊奶,冷藏,物理过程,酸奶发酵概况,酸奶的理化特性}

\begin{abstract}
对于新鲜,冷藏(4天/ 7°C)和冷冻/融化(1个月/ -18°C)的羊奶,评估了搅拌和均质化对酸奶生产的影响。结果表明,冷藏增加了发酵时间(> 90分钟),而来自冷冻/解冻牛奶的酸奶对后酸化更加敏感(pH降低高50%),并且硬度和粘合性降低(降低约25%)。搅拌和均质化改变了牛奶的结构,减少了发酵最初几个小时的天然奶油(10-100%),并增加了酸奶的质地参数。对牛奶进行发酵之前进行的物理过程的总体评估显示,冷藏和冷冻/融化的牛奶比新鲜牛奶具有更大的改良效果。从生产储存28天后,所有酸奶样品均未显示出脱水收缩作用,乳酸菌数量没有减少,粘附性和硬度也没有增加。因此,结果表明,牛奶冷藏和冷冻都会改变酸奶的发酵时间,缓冲能力和质地,并且搅拌和均质化由于脂肪和蛋白质之间的较高相互作用而改善了所有样品的酸奶质地。该信息对于羊奶生产商改善生产的酸奶的质量很有用,尤其是从先前冷藏的牛奶中获取酸奶时。
\end{abstract}

\section{介绍}
与牛奶相比,绵羊奶的固体含量高,因此主要用于制造乳制品(Tribst,Ribeiro,Leite Junior,de Oliveira和Cristianini,2018年)。它可以生产出一种奶油状且非常一致的酸奶(Güler-Akin,2005年),但产品的最终性能还取决于培养温度和所添加的培养物(Güler-Akin,2005年)。考虑到羊奶酸奶的营养价值和感官特性,人们对其的兴趣与日俱增(Zamberlin&Samaržija,2017年)。

由于动物生产率低,全年季节性生产牛奶和哺乳期短,因此获得羊奶产品受到限制。此外,在许多国家/地区,羊奶都是小规模生产,不允许大规模生产(Hilali,El-Mayda和Rischkowsky,2011;Tribst,Falcade和Oliveira,2019a)。因此,对于许多生产者而言,很难以对乳制品生产有利的方式获得牛奶。根据细菌总数和94/71 / EC指令(欧盟,1994年)规定的限值,冷藏时间为2.8天到9天不等,这取决于牛奶的初始质量,冷却速率和最终储存温度(Tribst等人,2019a)。冷冻是长时间储存牛奶的另一种选择。在这种情况下,重要的是要确保尽快冷却牛奶,最大程度地减少微生物的生长(Tribst等人,2019a ; Wendorff&Kalit,2017年)。

冷冻过程会影响牛奶成分的结构。主要的变化是由于脂肪小球的损伤,进而释放了膜的脂蛋白和降低了脂肪的稳定性(Tribst,Falcade,Ribeiro,LeiteJúnior,&Oliveira,2019b ; Wendorff&Kalit,2017)。这导致:(i)粒径增加,导致聚结和天然奶油分离(Pazzolla等,2013 ; Tribst等,2019b ; Zhang,Mustafa,Ng-Kwai-Hang,&Zhao,2006), (ii)脂质氧化速率的增加,和(iii)脂肪分解的发生(Zhang等,2006)。冻结还可能导致胶束蛋白不稳定,从而导致酪蛋白聚集(Muir,1984),由于多肽之间的氢键断裂,导致保水能力(WHC)降低(Wendorff,2001)。冷冻/解冻后的马苏里拉奶酪也观察到了类似的结果(Alinovi&Mucchetti,2020)。此外,可以观察到牛奶盐分平衡的变化,从而改变了可溶性钙的浓度(Kljajevic等人,2016 ; Tribst等人,2019b)并影响了牛奶的缓冲能力(Tribst等人,2019b)。即使强度低于冷冻强度,长时间在冷藏温度下储存也可能导致羊奶变化,沉淀和乳脂分离增加,这可能是由于蛋白酶和凝集素的作用(Tribst等人,2019b)。此外,尽管预期冷藏会导致胶体钙的溶解,如对牛奶观察到的(Fox,1997),但先前从冷藏羊乳获得的结果并未显示可溶性钙和胶体钙的百分比值发生变化(Tribst等人, 2019b)。

牛奶中的这些变化会降低乳制品的理化和微观结构稳定性。先前的结果表明,冷冻/解冻羊奶会增加发酵时间,并降低奶酪生产过程中的凝乳硬度(Pazzolla等,2013)。此外,冷冻会对酸奶的某些特性产生负面影响,减慢酸化速度,增加最终酸度和脱水收缩,并降低酸奶的稠度(Tribst等人,2018)。

搅拌和均质化等物理过程能够减少脂肪小球的大小,并促进酪蛋白胶束在其中的吸附/结合(Tribst等人,2020a)。主要结果是改善了乳剂中脂肪分散的稳定性,避免了结块的发生(Wendorff&Kalit,2017)。这可以最大程度地减少冷冻羊奶引起的问题,还可以提高羊奶的天然稳定性(Tribst等人,2020a)。该研究旨在比较发酵过程和从新鲜,冷藏和冷冻/解冻的羊奶中制得的酸奶的特性,这些羊奶经过均质处理(3.5 MPa)并在酸奶生产之前进行了搅拌(低剪切混合器)。

\section{材料与方法}
\subsection{牛奶与酸奶发酵过程}
Lacaune羊奶(3.8±0.1%脂肪,4.4±0.1%蛋白质,4.6±0.1%乳糖和13.8±0.2%固体含量)是通过混合30只健康动物(巴西的Porto Feliz)的牛奶而获得的。挤奶后,将样品分成三个等体积的10L。第一体积用于生产新鲜的酸奶(F)。第二个在7°C下保存4天,然后从冷藏牛奶中制得酸奶(R)。最后,将最后一个装在1 L高密度聚乙烯塑料袋中,并使用家用冷冻器在−18°C(\textasciitilde24 h)内缓慢冷冻,并在保存一个月后在7°C解冻。然后,该牛奶用于从冷冻/解冻的牛奶(FT)中生产酸奶。这些条件是根据Tribst等人的结果选择的。(2019a),并且在存储后,所有样本的计数均<3.0×105  CFU $mL^{-1}$。

将新鲜的,冷藏的和冷冻/解冻的牛奶样品分成3.3 L的三个新部分,然后将每个部分细分为三个新的部分(1.1 L),以获得一式三份的加工样品。将三份(1.1升)搅拌(使用家用搅拌器-B1000型,不列颠尼亚,巴西-1200 W低速搅拌2分钟),再进行三到低压均质化(3.5 MPa,使用高压均质机–型号为Panda Plus,意大利GEA-Niro-Soavi)和后三个未处理。在45℃下搅拌和均质化。

之后,将每个牛奶样品在水浴中进行热处理(90°C / 5分钟)。然后,将温度降至43℃和10 7  CFU $mL^{-1}$酸奶培养物(唾液链球菌亚种热和乳杆菌德氏亚种保加利亚,Y 450 B,萨科,意大利)中的溶液加入到该乳。将九个单独的灭菌锅(110 mL)和四个灭菌的硼硅酸盐玻璃管(20 mL)装满接种的牛奶,并在43°C下孵育。

通过测量pH下降(一式三份)和通过样品的反向散射光(%BS)进行观察,该样品通过880 nm的近红外(NIR)光扫描(Turbiscan MA2000,Formulation,法国),进行四次测量。使用Turbisoft 2.0软件对数据进行了分析(LeiteJúnior,Tribst和Cristianini,2017年)。

每隔45分钟(最多发酵4:30分钟)采集样品读数,然后每隔30分钟采集一次读数,直到每个酸奶样品的pH值达到4.60±0.05。最大乳化百分比(%Max Creaming)由试管中形成的乳膏线的高度与总样品高度之间的比例确定。如Tribst等人所述测定乳霜线。(2019b)。最大脱水收缩率(\%MaxMax Syneresis)由试管中形成的乳清线的高度与总样品高度之间的比例确定。当%BS比在试管中心测得的%BS低0.05%时,认为是脱水收缩线的起点,而在奶油线开始时,则认为该线的终点。

使用改良的Gompertz细菌生长方程式来适应发酵过程中的pH下降(Tribst等人,2018),以对所得结果进行建模(方程式(1))。发酵后,将酸奶样品在4°C冷藏。
\begin{equation}
		pH = pH_0 + (pH_\infty - pH_0) exp \left \{ -exp \left [ \frac{\mu e}{(pH_\infty - pH_0)} (-t) + 1 \right ] \right \}
\end{equation}
其中pH 0  =初始pH,pH∞  =最终pH,$\mu$=最大pH降低速率($h^{-1}$),$\lambda$=滞后相时间(h),t =时间(h)。

\subsection{酸奶中乳酸菌的数量和理化特性}
酸奶的乳酸菌(LAB)计数和理化特性在生产(4°C)储存0和28天后进行。LAB进行了三次重复(IDF,2003年)。为了进行质地测量,每个酸奶样品使用四锅。通过使用TA.XT2i型纹理分析仪(Stable Micro Systems,Godalming,英国),Texture Exponent 32软件(Stable Micro Systems)和5.0 kg的称重传感器通过单次压缩测试进行纹理分析。该测试包括硬度测量(在样品穿透过程中的正力峰)和粘附性(在拔出探针过程中的负力峰),并且以1 mm / s的测试速度以15 mm的穿透率进行测试使用直径25毫米,高度10毫米的圆柱形丙烯酸探针(Skryplonek等,2019)。样品的直径为5.2厘米,高度为5.3厘米。为避免样品之间的温度波动,将锅在7°C下保持至少3 h。立即将每个锅从其质地读数中移出,该读数在20°C的实验室中进行。总读取时间小于40秒。

此外,一式三份测定pH和可滴定酸度(AOAC International,1999)。在样品储存1天和28天后,使用发酵过程中所述的相同设备和条件,将在硼硅酸盐试管中发酵的样品用于测定%BS。使用在硼硅酸盐管中从酸奶高度10 mm到30 mm的酸奶高度(均质%BS的范围)收集的%BS数据,计算了保存1天和28天数据的%BS平均值。

\subsection{统计分析}
通过使用方差分析(ANOVA)评估了牛奶保存(新鲜,冷藏或冷冻/解冻的牛奶)和用于改善牛奶稳定性的物理过程(未加工,搅拌或均质化)对酸奶样品分析响应的影响。和Tukey检验在95%的α置信水平下进行。发酵1天和28天后,使用相同的测试比较相同的酸奶。皮尔逊检验相关性用于测量样品的%BS和%Max乳化之间的统计关系。使用XLSTAT软件(版本2015.2.02)(Addinsoft,巴黎,法国)进行分析。

\section{结果和讨论}
\subsection{酸奶的发酵特性}
图1显示了发酵期间每种酸奶获得的pH下降曲线。发酵时间受牛奶存储过程类型的影响,对于用新鲜牛奶生产的酸奶,该过程至少要快90分钟。另一方面,所研究的物理过程并未改变完成发酵所需的时间。所有获得的数据都很好地符合改进的Gompertz模型(0.96> $R^{2}$  > 0.99),表明样品的pH下降可以用公式(1)中的参数描述,如表1所示。

\begin{figure*}[!htb]
\centering
\includegraphics{imgs/img1}
		\caption{未经加工(A)或搅拌(B)或均质化(C)的新鲜,冷藏和冷冻/解冻羊奶生产的酸奶在发酵过程中pH下降。\\\includegraphics{imgs/blue}: 新鲜牛奶; \includegraphics{imgs/red}: 冷藏牛奶; \includegraphics{imgs/green}:冷冻/解冻的牛奶。虚线表示根据实验数据调整的Gompertz修改模型}
\end{figure*}

%表一
\begin{table*}[!htb]
\centering
\caption{修改后的Gompertz方程参数适用于pH降低的羊奶酸奶发酵曲线。}
\begin{threeparttable}
\begin{tabular}{lllll}
\hline
		参数 & 牛奶样本 & 未处理 & 搅拌牛奶 & 均质牛奶\\ 
\hline
		$ \lambda(h) $ & 新鲜的 & $ 2.04\pm0.15^{a, A} $ & $ 1.89\pm0.08^{b, B} $ & $ 2.11 \pm 0.20^{a, A} $\\
		& 冷藏的 & $ 1.76 \pm 0.01^{b, B} $ & $ 1.79 \pm 0.03^{b, B} $ & $ 1.99 \pm 0.04^{a, A}$\\
		& 冷冻的/解冻的 & $ 2.00 \pm 0.04^{a, B} $ & $ 2.08 \pm 0.04^{a, B}$ & $ 2.29 \pm 0.09^{a, A}$\\
		$ \mu(h^{-1}) $ & 新鲜的 & $ −0.73 \pm 0.02^{a, A} $ & $ −0.75 \pm 0.06^{a, A} $ & $ −0.75 \pm 0.07^{a, A} $\\
		& 冷藏的 & $ −0.50 \pm 0.02^{b, A} $ & $ −0.51 \pm 0.01^{b, A} $ & $ −0.56 \pm 0.01^{b, A} $\\
		& 冷冻的/解冻的 & $ −0.46 \pm 0.02^{b,B} $ & $ −0.50 \pm 0.02^{b,B} $ & $ −0.74 \pm 0.01^{a,A} $\\
		开始形成凝胶的pH范围 & 新鲜的 & 6.26–5.45 & 6.20–5.36 & 6.32–5.50\\
		& 冷藏的 & 6.19–5.62 & 6.19–5.62 & 6.20–5.65\\
		& 冷冻的/解冻的 & 6.29–5.78 & 6.28–5.73 & 6.40–5.60\\
		发酵3.5h后的pH & 新鲜的 & $ 5.50 \pm 0.07^{c,A} $ & $ 5.42 \pm 0.05^{b,A} $ & $ 5.55 \pm 0.05^{b,A}$\\
		& 冷藏的 & $ 5.65 \pm 0.06^{b,A} $ & $ 5.66 \pm 0.04^{a,A} $ & $ 5.71 \pm 0.06^{a,A} $\\
		& 冷冻的/解冻的 & $ 5.81 \pm 0.04^{a,A}$ & $ 5.77 \pm 0.05^{a,A} $ & $ 5.67 \pm 0.07^{a,B}$\\
\hline
\end{tabular}

\begin{tablenotes}
	\footnotesize
	\item[*] $ \lambda $=滞后相位时间(h)  %此处加入注释*信息
	\item[**] $ \mu $=最大pH下降速率($h^{-1}$)
	\item[a-b]小写上标指示通过Tukey检验评估的 牛奶在使用相同物理过程的不同牛奶样品生产的酸奶之间存在显着差异(p <0.05)。
	\item[b-B]大写字母上标表明 ,在相同的牛奶样品经过不同的物理处理后,通过Tukey检验评估的酸奶之间存在显着差异(p <0.05)。
\end{tablenotes}
\end{threeparttable}
\end{table*}

参数$\lambda$代表pH下降的滞后阶段(在LAB培养物产生能够引起pH降低的最小酸浓度之前花费的时间)。结果表明,未经加工的牛奶中,冷藏牛奶中的酸奶的滞后阶段比新鲜和FT(14%)稍低。比较搅拌样品,FT牛奶的$\lambda$稍高(<10%)。在均质样品之间,未观察到差异。此外,在鲜奶中采用不同的物理过程时,未观察到差异,而冷藏(13%)和冷冻/解冻的样品(29%)均化后的滞后阶段增加。

样品的初始pH在6.52和6.60之间,显示它们之间没有差异。因此,预计样品的初始酸度仅会发生很小的差异(Tribst,Falcade,LeiteJúnior和Oliveira,2020b)。新鲜和先前冷藏的牛奶之间在$\lambda$中观察到的差异很小(<15%),并且可以用初始缓冲能力,矿物质平衡和蛋白水解产生的肽存在的细微差异来解释,特别是对于冷藏样品(Tribst等人,2019b)。另一方面,就所进行的过程而言,均质化对$\lambda$的影响更大,这表明该过程由于矿物质平衡和蛋白质/脂肪构型的变化,特别是对于冷藏牛奶的变化,在高pH下增加了牛奶的缓冲能力。样本(Tribst等人,2020a)。

参数$\mu$是pH下降的最大速率(Tribst等,2018)。结果表明,用鲜牛奶制得的酸奶观察到更高的比率(\textasciitilde50%)(与物理过程无关),而用冷藏或冷冻/解冻的牛奶制得的酸奶观察到类似的行为。考虑到与新鲜牛奶相比,冷藏牛奶显示出更高的缓冲能力(pH 6.5-4.6)(Tribst等人,2019a,2020a),这是可以预期的。先前的结果还表明,与新鲜牛奶生产的酸奶相比,冷冻/解冻牛奶生产的酸奶的pH值降低幅度较小(Tribst等人,2018)。关于物理过程,均质化是唯一能够提高$\mu$的方法,并且仅适用于冷冻/解冻的牛奶。因此,数据表明,均质化能够恢复牛奶中冻结/解冻引起的物理变化,从而导致pH值介于6.5和4.6之间的缓冲能力降低(Tribst等人,2020a),从而导致相似的pH值分布从鲜牛奶中发酵酸奶的过程中减少。有趣的是,相同的均质样品显示出最高的$\lambda$值,这可能表明均质化改变了这些样品的缓冲特性,从而增加了高pH值牛奶的缓冲能力,而降低了低pH值牛奶的缓冲能力。

在酸奶发酵中,pH下降导致乳凝胶形成,这在酪蛋白的等电点最大(Lee&Lucey,2010)。反向散射光是间接测量凝胶形成的一种方法(LeiteJúnioret al。,2017),因为牛奶中蛋白质的聚集会导致%BS的增加(图2)。)。在发酵开始时,未经处理的样品的%BS比经过处理的样品低至少16%(P <0.05)。在通过搅拌或均质处理的样品之间未观察到差异,对于经过不同保存过程的样品(<2%),观察到一些微小差异。因此,物理过程引起了%BS的类似增加,可能与组成结构的改变有关,尤其是脂肪球尺寸的减小和蛋白质对小球的吸收更高(Tribst等人,2020a)。根据Holroyd(2013)由于悬浮液中脂肪球和酪蛋白胶束的大小和数量的变化,吸收和光散射会发生变化,因此牛奶是需要通过NIR研究的具有挑战性的基质。因此,必须将初始%BS视为每个样本的单独属性。因此,考虑到其特定的初始%BS,从每种酸奶中%BS的显着增加来确定聚集的开始。

\begin{figure*}[!htb]
\centering
\includegraphics{imgs/img2}
\caption{由未经加工(A)或经过搅拌(B)或均质化(C)的新鲜,冷藏和冷冻/解冻羊奶生产的酸奶在发酵过程中的反向散射曲线。\\\includegraphics{imgs/blue}:鲜牛奶;\includegraphics{imgs/red}:冷藏牛奶; \includegraphics{imgs/green}:冷冻/解冻的牛奶。}
\end{figure*}

根据%BS结果,计算开始凝胶形成的pH范围(表1)。尽管样品之间的pH降低速率不同,但所有样品的%BS首次升高发生在2h40min和3h30min之间(表1和图2)。根据这些结果,确定所有样品的凝胶形成都在pH 6.19和5.78之间开始。该pH范围高于开始在牛奶中形成凝胶所需的pH范围(pH 5.1- Lee&Lucey,2010年),这可以解释为考虑到羊奶中较高的蛋白质和固体含量,有利于在观察到之前形成蛋白质网络牛奶中的蛋白质(LeiteJúnior等人,2017)。

比较发酵3h30min后样品的%BS的Δ(表2),发现新鲜牛奶生产的酸奶比冷藏和冷冻/解冻牛奶生产的酸奶具有更高的值(P <0.05)(样品除外)用均质的冷冻/解冻牛奶生产)。可以解释这是由于在3小时30分钟时鲜牛奶中酸奶的pH值较低(pH值至少降低0.15 –表1),表明这些样品中的凝胶形成开始较早。此外,通常观察到物理过程会影响%BS的Δ,这对于用加工乳制得的样品来说很小。考虑到经过处理的和未经处理的样品(相同的牛奶罐头)在3h30min时的pH值相似(表1),这意味着由均质化和搅拌效应引起的%BS的初始增加降低了酸奶生产过程中%BS的增加速率。冷冻/解冻的酸奶的数据显示出不同的行为,未经加工和均质的牛奶生产的酸奶的%BS的Δ值较高(P> 0.05)。考虑到均质化能够提高样品的pH降低速率,从而导致与新鲜牛奶相似的聚集,可以解释这一点。

%表2
\begin{table*}
\centering
\caption{羊乳酸奶发酵过程中获得的近红外光反向散射参数。}
\begin{threeparttable}
\begin{tabular}{llllllll}		
	\hline
	牛奶样本 & 处理情况 & %BS-0小时 & %BS-3小时30分钟 & Δ%BS(3h30 – 0h) & %BS-最终 & %乳化最大值 & 最大脱水收缩率\\
	\hline
	新鲜的 & 未处理 & $49.5 \pm 0.1^{B,b}$ & $60.0 \pm 0.2^{B,a}$ & $10.6 \pm 0.2^{A,a}$ & $64.1 \pm 0.1^{C,a}$ & $11.7 \pm 1.9^{A,a}$ & $7.4 \pm 0.7^{A,a}$ \\
		   & 搅动的 & $59.3 \pm 1.9^{A,a}$ & $66.0 \pm 1.0^{A,a}$ & $6.8 \pm 1.0^{B,a}$ & $66.9 \pm 0.7^{B,a}$ & $9.5 \pm 3.8^{B,a}$ & $5.0 \pm 1.3^{B,b}$\\
		   & 均质的 & $58.5 \pm 0.2^{A,b}$ & $65.5 \pm 0.2^{A,a}$ & $7.0 \pm 0.3^{B,a}$ & $68.8 \pm 0.1^{A,a}$ & - & -\\
	冷藏的 & 未处理 & $49.9 \pm 0.3^{B,ab}$ & $57.1 \pm 0.6^{B,b}$ & $7.3 \pm 0.5^{A,b}$ & $63.7 \pm 0.3^{B,a}$ & $15.2 \pm 3.5^{A,a}$ & $5.5 \pm 0.7^{B,ab}$\\
		   & 搅动的 & $58.9 \pm 0.3^{A,a}$ & $62.8 \pm 0.8^{A,b}$ & $4.0 \pm 0.7^{B,b}$ & $67.9 \pm 1.5^{A,a}$ & $4.8 \pm 0.8^{B,ab}$ & $9.1 \pm 1.7^{A,a}$\\
           & 均质的 & $59.6 \pm 0.4^{A,a}$ & $62.7 \pm 0.4^{A,b}$ & $3.1 \pm 0.8^{B,b}$ & $3.1 \pm 0.8^{A,b}$ & - & $8.0 \pm 1.2^{A,a}$\\
	冷冻的/解冻的 & 未处理 & $50.0 \pm 0.2^{B,a}$ & $55.9 \pm 0.8^{C,c}$ & $5.9 \pm 1.0^{A,c}$ & $62.4 \pm 1.8^{B,a}$ & $11.3 \pm 0.9^{A,a}$ & $4.1 \pm 1.6^{B,b}$\\
		   & 搅动的 & $58.9 \pm 1.0^{A,a}$ & $62.1 \pm 1.5^{B,b}$ & $3.2 \pm 0.5^{B,b}$ & $64.9 \pm 2.1^{B,a}$ & $4.0 \pm 1.5^{B,b}$ & $5.9 \pm 1.7^{B,ab}$\\
		   & 均质的 & $58.1 \pm 0.4^{A,b}$ & $65.1 \pm 0.6^{A,a}$ & $7.0 \pm 0.3^{A,a}$ & $69.2 \pm 0.3^{A,a}$ & - & $9.9 \pm 1.4^{A,a}$\\
	\hline
\end{tabular}	
\begin{tablenotes}
	\footnotesize
	\item[A-B] 大写字母上标表明 ,在相同的牛奶样品经过不同的物理处理后,通过Tukey检验评估的酸奶之间存在显着差异(p <0.05)。
	\item[a-b] 小写上标指示通过Tukey检验评估的 牛奶在使用相同物理过程的不同牛奶样品生产的酸奶之间存在显着差异(p <0.05)。
	\item[-] 样品无乳脂或脱水收缩。
\end{tablenotes}
\end{threeparttable}
\end{table*}

根据图2,几乎所有样品的发酵4小时后,%BS读数都显示出渐近行为,这表明最大的聚集发生在pH 5.7和5.1之间(图1)。因此,与数据和牛奶相比(Lee&Lucey,2010),发酵过程中羊乳的聚集发生在较高的pH范围内。

用新鲜,冷藏和冷冻/解冻的牛奶生产的酸奶的最终%BS(表2)相似,而使用未加工(最低值),搅拌(对于新鲜和冷藏牛奶的酸奶的未加工样品)和均质牛奶生产的样品,最终%BS相似(在所有研究的条件下都高于未处理),并且这些差异始终低于11%。如最初的%BS所示,这种小的差异可能是由于样品中脂肪球的尺寸差异所致(Holroyd,2013年)。因此,如果酸奶浓度发生差异,则与物理过程中脂肪球的减少有关,而不是与不同牛奶样品发酵过程中形成的凝胶网络差异有关。

来自NIR反向散射的数据还确定了天然奶油的分离和脱水收缩的发生(表2)。奶油分离仅发生在未处理和搅拌的样品上。未经加工的样品形成了一层厚厚的奶油(11.3–15.2%),新鲜,冷藏和冷冻/融化的牛奶之间没有差异。另一方面,与新鲜牛奶相比,搅拌样品的奶油层减少了(4.0-9.5%),而冷冻/解冻牛奶的奶油层略小(P <0.05)。再有,所使用的不同物理过程比牛奶保存技术更重要。考虑到这些差异的原因是,由于应用的物理过程导致脂肪球直径较小以及蛋白质吸收到脂肪中的原因(Tribst等人,2020a)。此外,在聚集开始时就停止了乳化。这可以通过以下事实来解释:根据斯托克斯定律,增加样品的稠度会降低当前脂肪的迁移率。重要的是要考虑到,所用的羊奶脂肪含量低(3.8±0.1%),这可能是由于泌乳期,季节和饲喂的结果(Haenlein&Wendorff,2006年,第3章)。因此,对于脂肪含量为6%至7%的牛奶,最大程度地减少奶油的产生的挑战可能会更大,这表明我们的结果应谨慎用于脂肪奶。

另一方面,脱水收缩在聚集开始之后开始,由于蛋白质网络的收缩/相互作用,在发酵过程结束时达到最大值。几乎所有样品都发生脱水收缩(用新鲜均质牛奶生产的酸奶除外),并且受牛奶保存类型和所用物理过程的影响。对于新鲜牛奶,未加工样品的脱水收缩性较高(P <0.05)(比搅拌样品高50%),而对于冷藏牛奶生产的酸奶,脱水收缩性至少高45%(P <0.05) )用于搅拌(冷藏样品)和均质牛奶(冷藏和冷冻/解冻牛奶)制得的酸奶。

EL-扎哈尔,CHOBERT,Sitohy,Dalgalarrondo,和Haertlé(2003)的约15-20%α乳白蛋白,β乳球蛋白,所观察到的降解α小号-caseins和β酪蛋白在绵羊奶酸奶发酵的末端,其可以直接影响酸奶的脱水收缩。在研究的样本中,脱水收缩的差异可以通过乳凝胶中蛋白质网络强度的差异以及凝胶形成后剩余的WHC的差异来解释(Lee&Lucey,2010)。因此,据报道,陈酿的Mozzarela奶酪的WHC较高,这是因为蛋白质在部分降解后的水合作用范围增加了(Kou,Gunasekaran,Johnson,&Chen,2001)。与新鲜牛奶的酸奶脱水相比,类似现象的发生可能解释了用冷存储的未加工牛奶(预计其蛋白质降解水平更高)生产的样品的脱水收缩较低。另一方面,物理过程(搅拌或均质化)的引入使所观察到的冷库在脱水收缩中的影响逆转,从而加剧了这种情况。与完整蛋白质不同,冷冻/解冻的牛奶中发现的部分降解的蛋白质可能与尺寸减小的脂肪小球相互作用,从而大大降低了它们的保水能力。

酸奶发酵结果的总体评估表明,储存冷牛奶的主要影响是样品发酵所需的时间比新鲜样品发酵所需的时间更长。相反,由于搅拌和主要是均质化引起的结构性乳改性通常对发酵时间参数影响较小,但强烈影响形成的凝胶,具有更高的%BS值,减少的乳脂化和明显的脱水收缩。因此,结果表明,与乳保存方法无关,均质化和搅拌可用于改善羊乳酸奶的发酵特性。

\subsection{酸奶的LAB计数和理化特性}
结果表明,所有样品的LAB计数均保持> 8.69 log.$CFU^{-1}$(表3),并且在存储过程中计数没有明显减少,这是非常需要的(Tamine&Robinson,2007)。这表明,即使统计学上确定了显着差异(<0.3 log),其在微生物学上也不相关。

%表3
\begin{table*}[!htb]
\centering
\caption{生产后第1天和第28天,酸奶中的乳酸菌计数,pH,酸度和NIR散射百分比。}
\begin{threeparttable}
\begin{tabular}{lllllll}
	\hline
	储存(天)& 牛奶样本 & 处理情况 & LAB计数($log.CFU^{-1}$) & pH值 & 酸度(每克乳酸100克酸奶)& 近红外%反向散射\\ 
	\hline
	1 & 新鲜的 & 未处理 & $9.05 \pm 0.08^{A,a}$ & $4.44 \pm 0.03^{A,b}$ & $0.92 \pm 0.02^{A,a}$ & $63.83 \pm 0.13^{B,a}$\\
      &        & 搅动的 & $8.81 \pm 0.18^{A,a}$ & $4.34 \pm 0.01^{B,b}$ & $0.90 \pm 0.02^{A,a}$ & $67.81 \pm 0.84^{A,a}$\\
	  &        & 均质的 & $8.90 \pm 0.09^{A,b}$ & $4.43 \pm 0.03^{A,a}$ & $0.93 \pm 0.06^{A,ab}$ & $69.91 \pm 0.25^{A,a}$\\
	  & 冷藏的 & 未处理 & $9.08 \pm 0.06^{A,a}$ & $4.57 \pm 0.09^{A,a}$ & $0.87 \pm 0.03^{B,ab}$ & $63.79 \pm 0.50^{B,a}$\\
	  &        & 搅动的 & $9.14 \pm 0.29^{A,a}$ & $4.51 \pm 0.10^{AB,a}$ & $0.90 \pm 0.04^{AB,a}$ & $68.08 \pm 1.89^{A,a}$\\
      &        & 均质的 & $9.32 \pm 0.31^{A,a}$ & $4.42 \pm 0.08^{B,ab}$ & $0.93 \pm 0.04^{A,b}$ & $68.64 \pm 0.41^{A,a}$\\
      & 冷冻的/解冻的 & 未处理 & $8.91 \pm 0.09^{A,a}$ & $4.64 \pm 0.03^{A,a}$ & $0.85 \pm 0.01^{B,b}$ & $62.41 \pm 0.68^{C,a}$\\
      &               & 搅动的 & $9.01 \pm 0.12^{A,a}$ & $4.45 \pm 0.07^{B,ab}$ & $0.88 \pm 0.02^{B,a}$ & $66.75 \pm 1.49^{B,a}$\\
      &               & 均质的 & $9.07 \pm 0.07^{A,ab}$ & $4.33 \pm 0.03^{C,b}$ & $0.97 \pm 0.01^{A,a}$ & $70.41 \pm 0.31^{A,a}$\\
   28 & 新鲜的 & 未处理 & $8.97 \pm 0.29^{A,a}$ & $4.07 \pm 0.03^{A,a,*}$ & $1.08 \pm 0.03^{A,a,*}$ & $62.94 \pm 0.30^{C,a,*}$\\
      &        & 搅动的 & $9.05 \pm 0.13^{A,a}$ & $4.00 \pm 0.02^{B,a,*}$ & $1.11 \pm 0.01^{A,a,*}$ & $67.58 \pm 1.05^{B,a}$\\  
      &        & 均质的 & $8.99 \pm 0.21^{A,a}$ & $4.01 \pm 0.02^{AB,a,*}$ & $1.12 \pm 0.01^{A,a,*}$ & $69.96 \pm 0.69^{A,a}$\\
      & 冷藏的 & 未处理 & $8.96 \pm 0.16^{A,a}$ & $4.07 \pm 0.03^{A,a,*}$ & $1.01 \pm 0.03^{A,b,*}$ & $63.20 \pm 0.50^{B,a}$\\
      &        & 搅动的 & $9.04 \pm 0.18^{A,a}$ & $4.05 \pm 0.07^{A,a,*}$ & $1.05 \pm 0.03^{A,b,*}$ & $68.06 \pm 1.46^{A,a}$\\
      &        & 均质的 & $8.92 \pm 0.67^{A,a}$ & $4.03 \pm 0.05^{A,a,*}$ & $1.04 \pm 0.01^{A,b,*}$ & $69.94 \pm 0.42^{A,a,*}$\\
      & 冷冻的/解冻的 & 未处理 & $8.69 \pm 0.07^{B,b,*}$ & $3.91 \pm 0.03^{A,b,*}$ & $1.05 \pm 0.01^{B,b,*}$ & $62.45 \pm 0.80^{C,a}$\\
	  &               & 搅动的 & $8.99 \pm 0.08^{A,a}$ & $3.83 \pm 0.02^{AB,b,*}$ & $1.07 \pm 0.01^{AB,ab,*}$ & $67.98 \pm 0.78^{B,a}$\\
      &               & 均质的 & $8.86 \pm 0.18^{AB,a}$ & $3.77 \pm 0.03^{B,b,*}$ & $1.10 \pm 0.02^{A,a,*}$ & $70.72 \pm 0.35^{A,a}$\\
	\hline
\end{tabular}
\begin{tablenotes}
	\footnotesize
	\item[A-B] 大写字母上标表明,通过Tukey检验评估 的酸乳(p <0.05)在相同的牛奶样品经过不同的物理处理后生产的酸奶(储存时间1或28天)之间存在显着差异。
	\item[a-b] 小写上标表示通过Tukey测试评估 的酸乳(p <0.05)在经过相同物理过程的不同牛奶样品生产的酸奶(储存1或28天)之间存在显着差异。
	\item[*] 表示在储存1天和28天时评估的同一份酸奶样品之间通过Tukey测试评估的显着差异(p <0.05)。
\end{tablenotes}
\end{threeparttable}
\end{table*}

对于pH值,酸奶生产和冷却(1天)后立即得出的结果表明,pH在4.64到4.33之间,对于冷冻/解冻样品生产的酸奶,其pH值差异较大(P <0.05)(表3)。根据pH值,这些样品按从高到低的顺序排列,因为未处理>搅拌>均质化。此外,由于牛奶储存,观察到的差异很小(<4%),但差异显着,其中从新鲜牛奶中测定的酸奶的最低pH值(均质样品除外),与Katsiari,Voutsinas,和Kondyli(2002)。有趣的是,观察到具有较高μ值的样品的pH值较低(表1),这可能解释了发酵结束(pH在4.70到4.65之间)和冷却过程结束之间较高的pH下降。El-Zahar等人在发酵结束和冷却结束之间观察到类似的剧烈pH下降。(2003)。该下降与残留的LAB生长相关,表明酸奶培养物即使在非最佳温度下也能够继续生长并产生酸的能力。

酸奶的酸度(表3)与其pH值相适应,并且由新鲜和均质的羊奶制成的酸奶更酸。这有力地表明,最终pH的差异与样品中酸积累的差异有关,而不是由于其缓冲能力的关系。另一方面,样品的pH值和酸度的整体比较表明,某些pH值相似的样品具有不同的酸度(尤其是冷冻/融化的均质样品和新鲜搅拌的样品),表明缓冲能力的差异也可能会干扰样品的酸度。结果,但程度较小。

储存后,观察到酸奶的后酸化作用导致pH值下降(从-0.28到-0.74)(P <0.05)和酸度增加(从0.14到0.24%)(P <0.05)。新鲜和冷藏牛奶中的酸奶显示出相似的最终pH值(4.00–4.07),但两者之间的差异可忽略不计(<2%)。相反,用冷冻/解冻的酸奶制得的酸奶的pH值下降(至少高出0.15)。在冷冻/解冻的牛奶样品中,未加工牛奶制成的酸奶(0.73),搅拌(0.62)和均质(0.56)的酸奶观察到较高的pH下降。观察到的结果是这些样本之间的差异变平,仅在未处理和均质化的样本之间仍然显着。

储存的样品的酸度与pH值相矛盾,因为用新鲜和冷冻/解冻的样品制得的酸奶观察到了相似的高酸度值(P> 0.05),尽管新鲜牛奶中的酸度至少高0.2。另一方面,冷藏牛奶中的酸奶的酸度较低(P <0.05),与新鲜牛奶相比,其最终pH值相同。这些结果表明,在冷藏条件下,冷藏牛奶样品中的产酸量较低(5-8%)。此外,在后酸化pH值范围(约4.5至\textasciitilde3.8)中,来自鲜牛奶的酸奶比冷藏样品(主要是来自冷冻/解冻的牛奶)的缓冲能力更高。有趣的是,发酵曲线的结果表明,冷冻/融化的样品是在6.5至4.6之间具有最高缓冲能力的样品。这表明冷冻/解冻过程改变了羊乳的缓冲特性,在某些pH范围(pH 4.6以上)中增加了缓冲能力,而在其他pH范围(pH 4.6以下)中减少了缓冲能力。

用新鲜和冷冻/解冻的羊奶制成的酸奶的先前结果(Tribst等人,2018)与本研究中观察到的结果部分不同。尽管在两项研究中均观察到强烈的后酸化作用,达到了相似的最终酸度,但在先前的研究中观察到了更高的缓冲能力,最终pH值接近4.15。从Katsiari等人获得的结果。(2002)和El-Zahar等。(2003年)还表明,用新鲜羊奶生产的酸奶具有更高的缓冲能力(高乳酸/ pH值)。

考虑到羊奶中的总蛋白质含量在4.79%至6.10%之间,具体取决于喂养和哺乳期(Hilali等,2011),因此观察到的结果差异可以用蛋白质和固体含量(至少20与之前的样本相比,本研究样本的样本降低了%),从而降低了牛奶的缓冲能力(Kljajevic等,2016)。对于使用冷冻/解冻的牛奶生产的酸奶而言,这种差异更为关键,这可能是由于冷冻/解冻引起的盐分平衡变化,以及在搅拌或均质该牛奶时加剧的盐分平衡所致(Tribst等人,2020a)。

在第1天测得的平均%BS结果(表3)表明,未加工样品中的酸奶的数值始终较低(P <0.05),并且在酸奶生产之前用于羊奶储存的不同方法都不会改变该参数(P> 0.05)。此外,对于用新鲜和冷藏牛奶生产的酸奶,搅拌和均质过程观察到相似的结果,而对于冷冻/解冻的酸奶,均质促进%BS的增加(P <0.05)。比较发酵结束时(表2)和冷却酸奶后(表3)的%BS,样品中的%BS值有少量增加,可能表明酸奶的稠度有所提高(LeiteJúnior等,2017)。Nguyen,Afsar和Day(2018)生产的羊奶酸奶中观察到了类似的变化,这解释了酪蛋白颗粒在冷却步骤中溶胀,增加它们之间的接触面积并产生更坚硬的蛋白质的现象网络。

储存28天后,与第一天的值相比,仅新鲜的未经加工的牛奶和均质牛奶中的酸奶显示出%BS的微小差异(<2%)。相似的%BS在储存1天和28天后的响应几乎所有样品均表明,酸奶储存过程中的脱水收缩现象可忽略不计(这是由于样品在1和28天沿管高度的反向散射曲线重叠所证实的(数据未显示)。与牛奶相比,羊奶酸奶中形成的强大凝胶网络解释了其高持水能力和较低的脱水收缩(Nguyen等人,2018 ; Vianna等人,2019 ; Zamberlin&Samaržija,2017)。此外,根据El-Zahar等人的说法。(2003年)因此,蛋白酶的作用受到低温的限制,导致蛋白质的降解缓慢增加。这可能有助于解释为什么在酸奶的保质期内没有观察到额外的脱水收缩。

此外,由于搅拌和均质化可减小脂肪小球的粒径并增加脂肪与蛋白质之间的相互作用,从而使脂肪在酸性条件下陷于蛋白质网络中,因此可以预期在处理后的样品中观察到更高的%BS(Tribst等人,2020a)。考虑到%BS和%乳化最大数据之间的直接相关性,这些现象似乎是样品%BS不同的主要原因($R^{2}  = 0.78$)。

酸奶质地的硬度和粘合性参数如图3所示。有关生产和冷却后的酸奶的数据(第1天)显示,冷冻牛奶对两种质地参数均具有负面影响(P <0.05),但可以通过搅拌或均质过程将其回收,达到与从新鲜未加工牛奶中观察到的酸奶相同的值( P> 0.05)。此外,所研究的过程还能够从鲜奶(少量改进)和冷藏(更强烈的改进,特别是对于均质样品)的羊奶中改善(P <0.05)酸奶的参数。

\begin{figure*}
	\centering
	\includegraphics{imgs/img3}
		\caption{从新鲜的,冷藏的和冷冻/解冻的羊奶中进行搅拌或均质化后制得的酸奶的硬度(A)和粘合性(B)。\\\includegraphics{imgs/blue}:未处理的样品; \includegraphics{imgs/red}:搅拌样品; \includegraphics{imgs/green}:均质样品。ac小写上标表示通过Tukey测试评估的酸乳(p <0.05)在相同的牛奶样品(经过不同的物理过程)生产的酸奶(储存1天或28天)中存在显着差异。AC大写字母上标指示通过Tukey检验评估的酸乳(p <0.05)在使用相同物理过程的不同牛奶样品生产的酸奶(储存1或28天)中存在显着差异。*表示在存放1天(D1)和28天(D28)时,对同一酸奶样品进行的Tukey检验所评估的显着差异(p <0.05)。}
\end{figure*}

在储存期之后,观察到几乎所有样品的两个参数的值均增加(P> 0.05)。对于来自新鲜牛奶的酸奶,这些增加的样品在加工样品中比在对照样品中更显着,从而增加了两者之间的差异。对于由冷藏和冷冻/解冻的牛奶制得的酸奶,存在相反的效果,即从未经加工的奶制得的样品具有更大的增加,从而导致使用经加工和未经加工的奶制得的酸奶之间的差异减小。最后,比较过程的效果表明,用均质牛奶生产的酸奶(无论牛奶储存条件如何)仍然是最坚硬和粘性最大的样品。

坚硬被认为是酸奶最重要的质地特性(Vianna等,2019),与产品的口感和感官知觉有关(Nguyen等,2018)。在这种情况下,与牛奶或山羊奶制成的酸奶相比,羊奶酸奶具有更高的硬度和一致性(Hilali等人,2011;Zamberlin&Samaržija,2017)。这是由于羊奶中的蛋白质和脂肪含量较高,在发酵过程中形成了酪蛋白胶束聚集体,包裹着脂肪球和血清,形成了一个强大的三维蛋白质网络(Nguyen等人,2018年; Vianna等人,2019年))。这种结构的改变会影响酸奶的最终质地,从而解释了用加工样品(正变化)和冷冻/解冻牛奶(负变化)生产的酸奶所观察到的结果。

\section{结论}
长时间冷冻或冷冻/解冻的羊奶显示更长的发酵时间,以及获得的酸奶的特性变化。这些改变对于冷冻/解冻的牛奶而言是特别不希望的,这导致缓冲能力的降低,由于后酸化以及硬度和粘合性的损失,导致pH值降低。

另一方面,所研究的物理过程改变了牛奶,进而改变了酸奶的质地,尽管它对发酵时间和所生产酸奶的理化参数没有影响或几乎可以忽略不计。由于在牛奶的两个过程中,脂肪和蛋白质之间的相互作用更大,因此在搅拌的第一个小时中观察到的搅拌牛奶(主要是均质牛奶)的硬度和粘附性的增加与乳脂减少的发生直接相关。此外,尽管搅拌和均质化能够改善从新鲜的,冷藏的和冷冻/解冻的牛奶中获得的酸奶,但是对于冷冻牛奶酸奶观察到更大的效果。这些数据对于羊奶生产商改善所生产酸奶的质量很有用。特别是那些从以前冷藏的牛奶中获得的牛奶。

\section*{CRediT作者贡献声明}
\textbf{Alline Artigiani Lima Tribst:}概念化,方法论,形式分析,写作-原始草案,监督,项目管理,资金获取。\textbf{Luiza Toledo Piza Falcade:}调查,形式分析,可视化。\textbf{NatháliaSilva Carvalho:}调查,形式分析,可视化。\textbf{Bruno Ricardo de Castro Leite Junior:}方法论,形式分析,写作-原始草案。\textbf{Miguel Meirelles de Oliveira:}概念化,方法论,形式分析,写作-原始草案。

\section*{竞争利益声明}
我谨代表所有共同作者声明,我们与其他作者或机构没有利益冲突。

\section*{致谢}
作者要感谢圣保罗研究基金会(FAPESP)的财政支持(项目号2017 / 02832-4)和里马小企业的牛奶捐赠。

\section*{参考文献}
\small
Alinovi, M., \& Mucchetti, G. (2020). Effect of freezing and thawing processes on high-
moisture Mozzarella cheese rheological and physical properties. LWT, 124, 109–137.

Aoac International (1999). Official methods of analysis (16th ed.). Washington, DC: AOAC
Int.

El-Zahar, K., Chobert, J.-M., Sitohy, M., Dalgalarrondo, M., \& Haertlé, T. (2003).
Proteolytic degradation of Ewe milk proteins during fermentation of yoghurts and storage. Nahrung-Food, 47(3), 199–206.

European Union (1994). Directive 94/71/EC. Directive amending Directive 92/46/EC
laying down the health rules for the production and placing on the market of raw
milk, heat-treated milk and milk-based products. Official Journal of. European
Community, L368, 33–37.

Fox, P. F. (1997). Advanced dairy chemistry: Lactose, water, salts and vitamins (2nd ed.).
London: Chapman \& Hall536p.

Güler-Akin, M. B. (2005). The effects of different incubation temperatures on the acet-
aldehyde content and viable bacteria counts of bio-yogurt made from Ewe's milk.
International Journal of Dairy Technology, 58(3), 174–179.

Haenlein, G. F. W., \& Wendorff, W. L. (2006). Sheep milk. In Y. W. Park, \& G. F. W.
Haenlein (Eds.). Handbook of milk of non-Bovine mammals (pp. 137–191). (1st ed.).
New Jersey: Blackwell Publisher.

Hilali, M., El-Mayda, E., \& Rischkowsky, B. (2011). Characteristics and utilization of
sheep and goat milk in the Middle East. Small Ruminant Research, 101(1–3), 92–101.

Holroyd, S. E. (2013). Review: The use of near infrared spectroscopy on milk and milk
products. Journal of Near Infrared Spectroscopy, 21, 311–322.

IDF (International Dairy Federation) (2003). Yoghurt/enumeration of characteristic micro-
organisms – Colony count technique at 37 °C. Standard no 117. Brussels, Belgium: IDF.

Katsiari, M. C., Voutsinas, L. P., \& Kondyli, E. (2002). Manufacture of yoghurt from stored
frozen sheep milk. Food Chemistry, 77, 413–420.

Kljajevic, N. V., Jovanovic, S. T., Miloradovic, Z. N., Macej, O. D., Vucic, T. R., \&
Zdravkovic, I. R. (2016). Influence of the frozen storage period on the coagulation
properties of caprine milk. International Dairy Journal, 58, 36–38.

Kou, M. I., Gunasekaran, S., Johnson, M., \& Chen, C. (2001). Nuclear magnetic resonance
study of water mobility in Pasta Filata and non-Pasta Filata mozzarella. Journal of
Dairy Science, 84, 1950–1958.

Lee, W. J., \& Lucey, J. A. (2010). Formation and physical properties of yogurt. Asian-
Australasian Journal of Animal Sciences, 23(9), 1127–1136.

Leite Júnior, B. R. D. C., Tribst, A. A. L., \& Cristianini, M. (2017). Comparative study
among rheological, near-infrared light backscattering and confocal microscopy
methodologies in enzymatic milk coagulation: Impact of different enzyme and pro-
tein concentrations. Food Hydrocolloids, 62, 73–82.

Muir, D. D. (1984). Reviews of the progress of dairy science: Frozen concentrated milk.
Journal of Dairy Research, 51(4), 649–664.

Nguyen, H. T. H., Afsar, S., \& Day, L. (2018). Differences in the microstructure and
rheological properties of low-fat yoghurts from goat, sheep and cow milk. Food
Research International, 108, 423–429.

Pazzolla, M., Dettori, M. L., Manca, F., Noce, A., Piras, G., Pira, E., et al. (2013). The effect
of long-term freezing on renneting properties of sarda sheep milk. Agriculturae
Conspectus Scientificus, 78(3), 275–279.

Skryplonek, K., Henriques, M., Gomes, D., Viegad, J., Fonseca, C., Pereira, C., et al.
(2019). Characteristics of lactose-free frozen yogurt with κ-carrageenan and corn
starch as stabilizers. Journal of Dairy Science, 102(9), 7838–7848.

Tamine, A. Y., \& Robinson, R. K. (2007). Yogurt science and technology (2nd ed.). Boca
Raton: CRC.

Tribst, A. A. L., Falcade, L. T. P., Carvalho, N. S., Cristianini, M., Leite Júnior, B. R. C., \&
Oliveira, M. M. D. (2020a). Using physical processes to improve physicochemical and
structural characteristics of fresh and frozen/thawed sheep milk. Innovative Food
Science \& Emerging Technology, 59, 1–11 article number 102247.

Tribst, A. A. L., Falcade, L. T. P., Leite Júnior, B. R. C., \& Oliveira, M. M. D. (2020b). Why
are most physicochemical parameters not useful in predicting the quality of sheep
milk? International Journal of Dairy Technology, 73(1), 292–295.

Tribst, A. A. L., Falcade, L. T. P., \& Oliveira, M. M. D. (2019a). Strategies for raw sheep
milk storage in smallholdings: Effect of freezing or long-term refrigerated storage on
microbial growth. Journal of Dairy Science, 102(6), 4960–4971.

Tribst, A. A. L., Falcade, L. T. P., Ribeiro, L. R., Leite Júnior, B. R. D. C., \& Oliveira, M. M.
D. (2019b). Impact of extended refrigerated storage and freezing/thawing storage
combination on physicochemical and microstructural characteristics of raw whole
and skimmed sheep milk. International Dairy Journal, 94, 29–37.

Tribst, A. A. L., Ribeiro, L. R., Leite Junior, B. R. D. C., de Oliveira, M. M., \& Cristianini, M.
(2018). Fermentation profile and characteristics of yoghurt manufactured from
frozen sheep milk. International Dairy Journal, 78, 36–45.

Vianna, F. S., Canto, A. C. V. C. S., Lima, B. C., Salim, A. P., Balthazar, C. F., Panzenhagen,
P., et al. (2019). Milk from different species on physicochemical and microstructural
yoghurt properties. Ciência Rural, 49(6), e20180522.

Wendorff, W. L. (2001). Freezing qualities of raw ovine milk for further processing.
Journal of Dairy Science, 84, E74–E78.

Wendorff, W. L., \& Kalit, S. (2017). Processing of sheep milk. In Y. W. Park, G. F. W.
Haenlein, \& W. L. Wendorff (Eds.). Handbook of milk of non‐Bovine mammals (pp. 210–
221). New Jersey: John Wiley \& Sons.

Zamberlin, Š., \& Samaržija, D. (2017). The effect of non-standard heat treatment of
sheep's milk on physico-chemical properties, sensory characteristics, and the bac-
terial viability of classical and probiotic yogurt. Food Chemistry, 225, 62–68.

Zhang, R. H., Mustafa, A. F., Ng-Kwai-Hang, K. F., \& Zhao, X. (2006). Effects of freezing
on composition and fatty acid profiles of sheep milk and cheese. Small Ruminant
Research, 64, 203–210.

\end{document}
