% 海洋生物技术论文
\documentclass[a4paper,twocolumn]{ctexart}
\usepackage{authblk}
\usepackage{geometry}
\usepackage{cite}
%\usepackage[document]{ragged2e}

\geometry{left=1.8cm,right=1.8cm,top=2cm,bottom=2cm}
\setlength{\columnsep}{25pt}

\title{海洋生物技术的应用领域}
\author[1]{彭王震\thanks{Corresponding author.\\E-mail address: pwz113436@gmail.com}}
\affil[1]{Ocean University of China, Qingdao, Shandong, China}
\date{}

\begin{document}
\ctexset{section = {format={\Large \bfseries }}}
\maketitle  %标题和作者

%%begin-------------------中英文摘要-----------------------%%
\twocolumn[
	%twocolumn: 双栏article下的单栏摘要
	\begin{@twocolumnfalse}
  	\renewcommand{\abstractname} {} %不显示摘要名字
	\begin{abstract}
	\vspace{-3em}
	%vspace:调整垂直空白,可以自己调整;缩小abstract和center(以及maketitle)的间距
	%\noindent %备用:摘要无缩进
	{\bf 摘{} 要:}
	{\small 我国拥有广阔的海洋领土, 习近平总书记说过“海洋是高质量发展战略要地。要加快海洋科技创新步伐, 提高海洋资源开发能力, 培育壮大海洋战略性新兴产业”, 同时我国也是世界上最大的水产养殖国。经过多年的发展和积累, 我们国家已具备较高水平的海洋生物技术,这些新技术包括海洋基因工程、海洋细胞工程和海洋蛋白质工程等。并且,很多技术在水产品育种、水产品病害防治和海洋产品开发等领域彰显着重要的优势。}
	\par%空的新行的高度。
	\textbf{关键词}:海洋生物技术, 基因工程, 细胞工程,转基因技术, 病害防治, 水产品育种
	\vspace{2em}
	\end{abstract}
  	\renewcommand{\abstractname} {} %不显示摘要名字
	\begin{abstract}
	\vspace{-3em}
	%vspace:调整垂直空白,可以自己调整;缩小abstract和center(以及maketitle)的间距
	%\noindent %备用:摘要无缩进
	{\bf Abstract:}
	{\small my country has a vast maritime territory. General Secretary Xi Jinping said that “the ocean is a strategic place for high-quality development. It is necessary to accelerate the pace of marine scientific and technological innovation, improve marine resource development capabilities, and cultivate and strengthen marine strategic emerging industries.” At the same time, my country is also the largest Aquaculture country. After years of development and accumulation, our country has possessed a relatively high level of marine biotechnology. These new technologies include marine genetic engineering, marine cell engineering and marine protein engineering. In addition, many technologies have significant advantages in the fields of aquatic product breeding, aquatic product disease prevention, and marine product development.} 
	\par%空的新行的高度。
	\textbf{Keywords}:Marine biotechnology, genetic engineering, cell engineering, Genetically modified technology, disease control, aquatic product breeding
	\vspace{2em}
	\end{abstract}
	\end{@twocolumnfalse}
]
%%end-------------------中英文摘要-----------------------%%

海洋是地球上潜力最大的资源库,它不仅能提供人类需要的优质蛋白质,还含有丰富的生物活性物质,是解决人类所面临的食物、资源和环境三大难题的最佳出处。为了开发海洋特别是开发海洋生物资源,我们发明很多先进的海洋生物技术。这些技术是运用海洋生物学与工程学的原理和方法,利用海洋生物或生物代谢过程生产有用的生物制品或定向改良海洋生物遗传特性的综合性科学技术。现如今,这些技术在很多领域有着重要的应用。

\section{海洋生物技术运用于水产育种}
我国自古以来就有使用水产品的习惯,但在生产力低下的封建朝代,没有形成大规模的养殖。我国水产品的大规模养殖真正开始于近现代,但起步阶段面临着很多问题,育种问题就是其中之一。海洋生物技术的应用,为水产育种增添了活力,大大提高了我国水产品的产量。还有,引入海洋生物技术, 既能够增加养殖品种,又可以提高养殖质量, 水产养殖具有的科技含量和经济效益也得到了不同程度的优化\cite{祝晓栋2019探究海洋生物技术在水产养殖中应用的可行性}。

根据相关资料显示,目前的应用有转基因技术培育和改良新品种、细胞工程来调控生物性别和藻类细胞育苗等。

\subsection{转基因技术培育和改良新品种}
随着人类对海洋自然资源的不断捕捞,水产自然资源已经越来越匮乏,世界水产养殖产量
在水产食物中所占的比例从2002年的29.9\%增加到2018年的46\%。再加上人们生活水平的提高,人们对水产品的质量和产量有着更高的要求。在这样的背景下,利用转基因技术对水产养殖动物进行改良是提高水产动物产量和质量的一项根本保障。

转基因技术涉及到DNA重组技术、细胞培养技术等多学科的交叉和融合,培育转基因水产动物,首先就是要获取外源目的基因并导入受体细胞,常用的水产动物转基因技术有显微注射、电穿孔法、基因枪轰击法、精子介导法等\cite{黄海燕2009转基因技术在水产动物中的运用}。

在水产动物中,最早开展的是鱼类的转基因技术,1985年,中国科学院水生生物研究所朱作言所在的学科报道了世界上第一例转基因鱼,也是世界上第一例转基因水产动物\cite{1985Novel}。随后的实验证明,这种导入了人类生长激素基因的金鱼生长速率明显高于没有转基因的普通金鱼。此外,有团队培育了转人类生长激素的泥鳅,发现这种泥鳅在135天内比非转基因对照组泥鳅生长速度快3~4.6倍\cite{朱作言1986BIOLOGICAL}。在后续的研究中,陆续有鲑鱼、罗非鱼、草鱼、斑马鱼、鲤鱼等不同鱼类获得不同类型的转基因鱼。在转基因鱼苗的研究中,最早关注的基因就是生长激素基因。它的目的就是改良经济性状,提高产量和生产效益,已经成功地从鲑鱼中提取生长激素基因并导入罗非鱼中,在人巨细胞病毒(human cytomegalovirus,CMV)启动子的驱动下,培育的 “全鱼转基因鱼” 消耗的食物比对照组少 3.6 倍,食物转化效率提高了 2.9倍\cite{2000Growth}。当然,除了生长激素基因的注入。将抗冻蛋白的基因转入水产动物也是目前的研究热点之一,抗冻蛋白可以使一些高纬度鱼类的体液凝固点降低至-1.7°C甚至-2°C,能大大提高生物对低温的适应性,提高其对低温的抵抗力,从而扩大养殖范围。

除了鱼类的转基因育种,在虾和软体动物中也有运用到转基因技术。Powers 等 1995 年首次通过电穿孔方法将外源的鲑的生长素基因成功转入红鲍(Haliotis rufescens)的胚胎,培育出可快速生长的鲍, 这是第一例外源 DNA 成功转入贝类胚胎中\cite{2010Biotechnology}的报道。Buchanan等(2001)报道了美洲牡蛎(C. virginica)转基因研究,采用电穿孔和化学介导法将氨基糖苷磷酸转移酶(aminoglycoside phosphotransferase II)基因成功地转入美洲牡蛎幼虫,显著提高了转基因牡蛎对抗生素G418 的耐受性。

\subsection{细胞工程调控生物性别}
在许多鱼类中,一种性别比另一种性别生长得更快或成熟得更早,这些差异在水产养殖条件下比较明显。所以,养殖者为了获得更加可观的经济效益,或选择具有较快生长速度和较大体形的性别种群。例如在罗非鱼中,雄鱼比雌鱼生长快50\%,养殖户都希望养雄鱼,为了得到雄鱼,西南大学淡水鱼类资源与生殖发育教育部重点实验室联合美日科研人员先对雄鱼(XY)进行雌激素诱导,使其雌性化,产出鱼卵,再用普通雄鱼使其受精,选育出“超雄鱼”(YY),后用超雄鱼对雌鱼(XX)受精,就得到全雄鱼(XY)。这意味这使用细胞工程技术可以使罗非鱼性别可控。

\subsection{细胞工程进行藻类育苗}
海洋藻类养殖是我国水产作物养殖的一个重要部分,特别是海带的产量多年位居世界前列。但是常规的自然育苗方法周期长、能耗大,运用海洋生物技术可以有效的解决这些问题。由中国科学院海洋研究所、上海水产大学和青岛海洋大学承担的国家 “七五” 攻关课题 “紫菜体细胞育苗” 于 1990年通过验收。专家认为,该课题利用细胞培养技术进行育苗,比传统方法大大缩短育苗时间,育苗室占地面积小,成本低,具有一定的社会和经济效益,达到国际领先水平\cite{王素娟1986坛紫菜原生质体的超微结构观察}。在90年代,裙带菜和海带单配体克隆育种育苗取得成功,为经济褐藻良种选育和幼苗培育以及我国海洋产业的迅速发展开拓了新途径\cite{王美功1999开创海藻育种育苗的新路子——访中科院海洋研究所研究员吴超元}。

\section{海洋生物技术运用于病害防治}
有数据表明,自21世纪初以来,水产养殖生物疾病的发病率一直处于直线上升阶段,尤其是病毒性疾病的爆发和扩散,这一因素对海水养殖的发展产生了巨大的阻碍作用。同时,海洋生物技术研究在疾病预防和控制中的应用引起了广泛的关注。

水产养殖中出现病害的原因有很多,其中养殖生态环境遭到破坏是一个方面。如果修复养殖生态环境,将有益于水产品的生长和发育。修复过程就运用了海洋生物技术,应用较广的一些天然微生物制剂,既有单一菌株的制剂如光合细菌等,又有复合菌株的制剂如芽孢杆菌类制剂和益生素等。这些生物制剂能够降低养殖环境中的氨氮、硫化氢、亚硝酸盐,提高溶解氧,对养殖环境有着一定程度的修复作用。

当然,除了修复被污染的养殖环境,直接的办法就是控制细菌性病原体,传统的办法是使用抗生素,但是大量使用抗生素容易使生物产生耐药性,而且药物残留的问题也无法得到有效的解决。因此,生物技术是一个很好的突破口。常用的生物防治技术有反义技术、阻断病毒传播、研发基因工程疫苗等。

反义技术常被用来阻断病毒功能发挥,已有研究表明反义 RNA 可以成功抑制真核细胞、原核细胞中的多种病毒基因。常见的病毒传播方式有水平传播和垂直传播两种。水平传播指的是带毒生物、带毒物质在水体环境中感染健康个体。阻断水平传播的方式, 主要是利用海洋生物技术, 检测饵料安全、水域环境,如果发现有病毒存在, 及时采取隔离措施, 避免病情进一步恶化。垂直传播指的是亲本传播子代。阻断垂直传播的方式包括核算探针、单克隆抗体等。养殖者可以根据检测结果,选择无病原亲本完成育苗\cite{奚根金2015水稻栽培技术研究}。基因工程疫苗指的是在病毒、细菌中,对能够免疫保护的抗原基因进行分离,将其与载体进行结合,在菌株上表达并扩增重组 DNA,最终制得抗原蛋白质。与常规疫苗相比,它具有高效性和转移性。

\section{海洋生物技术运用于产品开发}
近几年来,利用海洋生物技术开发新的药物、新的酶、新的保健品已经屡见不鲜。特别的,生物技术在海洋微生物和海洋微藻生物活性物质的研究开发方面发迅速,已取得不少成果。如利用发酵法快速生产海洋微藻,并从中提取$\omega$-3高度不饱和脂肪酸、藻胆蛋白、维生素C;以水华束丝藻制取石房蛤毒素,以海洋细菌生产河豚毒素等。应用生物培养、基因工程、DNA重组、发酵工程等生物技术已经成了推进海洋药物产业化的首选技术\cite{黄美珍2002海洋药物开发与生物技术}。由甲壳动物所提炼出的甲壳素, 具有十分突出的药用功能, 可以被制造成绷带、缝线、人工皮肤等。此外, 海洋生物技术还可以被用来开发肥料、饲料、功能食品或轻化工产品, 例如以鱼油、海藻多糖为原料的功能食品。

中国海洋大学一直致力于海洋药物开发,利用先进的生物技术开发了两款世界级海洋药物,一个是藻酸双酯钠片,主要用于缺血性脑血管病如脑血栓、脑栓塞、短暂性脑缺血发作及心血管疾病如高脂蛋白血症、冠心病等疾病的防治,也可用于治疗弥漫性血管内凝血、慢性肾小球肾炎及出血热等;一个是甘露特纳(GV-971),可用于轻度至中度阿尔茨海默症,改善患者认知功能。

\section*{结束语}
海洋生物技术目前的应用领域非常多,从培育新的水产物种到生产海洋药物,从选育高产的海藻作物到开发新的海洋食品,海洋生物技术已经深深地渗透到我们生活和生产的方方面面。在未来的发展中,随着人类对海洋资源的进一步开发和利用,海洋生物技术会得到更好的提升与拓展。

\bibliographystyle{IEEEtran}
\bibliography{main.bib}

\end{document}	
